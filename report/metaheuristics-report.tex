\documentclass[a4paper,10pt]{report}
\usepackage[utf8]{inputenc}

% Title Page
\title{Progressive Party Problem}
\author{Adrien DROGUET}


\begin{document}
\maketitle

%\tableofcontents
%\pagebreak

%\begin{abstract}
%\end{abstract}

\chapter{Progressive Party Problem}
\section{Description}
L'équipage de $X$ yachts doit se rencontrer au fil de $n$ périodes. Certains équipages
vont servir d'hôtes pour d'autres au fil des périodes.

\subsection{Contraintes de validité}
De manière réaliste, on équipage ne peut à la fois être hôte et être accueilli par un autre équipage,
une telle solution devrait être refusée d'office. Cependant, par souci de simplicité cette contrainte
n'est pas vérifiée au-delà de la solution initiale.


\subsection{Contraintes de qualité et calcul des coûts}
\subsubsection{Contrainte 1}
Chaque équipage doit changer d'hôte à chaque période. Pour une période $n$, chaque équipage ayant le
même hôte qu'à la période $n-1$ augmente le coût de 1. Note : on interprète cela comme signifiant
qu'un équipage donné ne devrait pas être hôte pour deux périodes consécutives.

\subsubsection{Contrainte 2}
Une paire d'équipage $x$ et $y$ ne doivent se rencontrer au plus qu'une seule fois. Pour un équipage
$x$, le coût est calculé en comptant le nombre d'équipages figurant plus d'une fois dans la liste
des équipages rencontrés de la dernière période.

\subsubsection{Contrainte 3}
Un bateau ne peut contenir plus que sa capacité d'accueil maximale pour une période donnée. Chaque infraction
entraîne un coût de 1. Note : l'augmentation du coût n'est pas fonction de l'ampleur du dépassement 
de capacité.


\pagebreak
\chapter{Développement}
\section{Structure du programme}
\subsection{Représentation d'une solution}
\paragraph{}
Notre objet \textbf{Solution} consiste principalement en une série de vecteurs de taille $p$,
avec $p$ = nombre de périodes, contenant des tableaux associatifs représentant les différents
équipages et les données qui leurs sont associés :
\begin{itemize}
	\item hôte de la période actuelle (0 si l'équipage reste sur son propre bateau)
	\item taux d'occupation du bateau pendant la période actuelle (sans compter l'équipage lui-même).
	\item liste des équipage rencontrés jusqu'à la période actuelle.
	\item l'ensemble des hôtes de la période actuelle.
	\begin{itemize}
		\item Note : ce champs devait être utilisé pour déterminer rapidement si un équipage est
		hôte ou non, et aider à maintenir la contrainte de validité.
	\end{itemize}
\end{itemize}

\paragraph{}
La classe \textbf{Solution} contient également nos méthodes de calcul de coût, ainsi qu'une liste
des objets bateau (\textbf{Boat}) de notre système.

\subsection{Modélisation des heuristiques}
Chaque heuristique étant dérivée de la précédente, elles font tout partie de la même famille de classe,
avec \textbf{Heuristic} (classe abstraite) $\to$ \textbf{LocalDescent} $\to$ \textbf{LocalSearch} $\to$
\textbf{TabuSearch}, en ne re-implémentant que les méthodes variant d'une heuristique à l'autre.

\pagebreak
\section{Heuristiques implémentées}
Note: Les résultats d'exécution sont détaillés en fin de rapport.
\subsection{Descente best fit}
\paragraph{}
La première heuristique implémentée est une exploration descendante type \textit{Best Fit}: pour un voisinage donné, on explore toutes les solutions possibles et on retient la plus améliorante.

\paragraph{}
Cette approche est coûteuse et a tendance à nous entraîner rapidement dans un optimum local.


\subsection{Recherche locale}
\paragraph{}
La deuxième approche est dérivée de la première : il s'agit d'une exploration descendante de type \textit{First Fit}. On effectue des mouvements aléatoires en ne retenant que les améliorants.

\paragraph{}
On obtient de meilleurs résultats plus rapidement qu'avec l'heuristique précédente.

\subsection{Recherche tabou}
\paragraph{}
L'approche finale est dérivée de la précédente : on explore de la même manière, mais en gardant un 
historique des changements effectués. Lorsque l'on atteint un optimum local, on remonte dans la pile
des changements effectués et on ne s'arrête que lorsque l'on a "épuisé" l'historique.

\paragraph{}
\textbf{\textit{Note importante :}} Taille maximale de l'historique déterminé à la compilation dans
$tabuSearch.hpp$ par \textit{\#define HISTORY\_SIZE}.

\paragraph{}
Avec un historique de taille faible (5), on obtient des coûts et des temps de calcul similaires à ceux
de la recherche locale first fit standard.

\paragraph{}
On ne constate pas d'amélioration flagrante des coûts finaux obtenus par rapport à la recherche locale
type first fit, même en faisant varier la taille de l'historique. Note: comparativement moins d'essais
on été réalisés pour cette heuristique, il est donc possible qu'elle n'ait pas été suffisamment
raffinée pour être efficace.



\section{Voisinages considérés}
\subsection{One Move}
On change l'hôte d'un équipage pour une période donnée. Note : l'équipage peut choisir de devenir hôte.

\subsection{Swap}
On choisit deux équipages pour une période donnée, et on échange leur hôte.

\section{Choix des équipages cibles}
\subsection{Choix aléatoire}
On choisit de manière complètement aléatoire un équipage autre que celui pour lequel on doit choisir
une cible.

\subsection{Plus grand conflit}
Cette approche consiste à calculer l'équipage engendrant le plus grand nombre de conflits pour une
période donnée, de manière à concentrer nos efforts sur les équipages ayant le plus à gagner.

\paragraph{}
Quelques implémentations basiques ont été réalisées, mais le temps manquant pour raffiner ce processus
cette approche a été abandonnée au profit du choix aléatoire.


\pagebreak
\section{Résultats}
Notes: 
\begin{itemize}
	\item On ne teste pas nos heuristiques sur la même instance de départ, la significativité
	du coût final est donc douteuse.
	\item taux d'amélioration = coût initial / coût final
	\item h = taille de l'historique de Tabu Search
\end{itemize}

\begin{figure}[h]
	\begin{tabular}[H]{| l | c | c | c | c |}
		\hline
		 & Local Descent & Local Search & Tabu Search (h = 5) &  Tabu Search (h = 30) \\
		\hline
		Coût final & 299 & 280 & 277 & 285 \\
		\hline
		Taux d'amélioration & 7.5 & 14.7 & 14.6 & 15.5 \\
		\hline
		Écart type & 2 & 3.9 & 3.2 & 3.5\\
		\hline
	\end{tabular}
	\caption{Compilation des résultats de 15 lancements}
\end{figure}


\end{document}
